
\documentclass{elsarticle}

\usepackage{lineno,hyperref}
\usepackage{amsmath}
\usepackage{mathptmx}
\usepackage{graphics}
\usepackage{setspace}\doublespacing
\modulolinenumbers[5]

\journal{Journal of Sports Sciences}

\bibliographystyle{model2-names.bst}\biboptions{authoryear}

%% Start your document body here
\begin{document}

\begin{frontmatter}
  \title{An objective assessment of point scoring systems in Formula 1 motor racing}
  \author{Robin K. S. Hankin\fnref{fn1}}
  \address{AUT University, 55 Wellesley Street, Auckland 1010, New Zealand\\
  }

\fntext[fn1]{\url{https://orcid.org/0000-0001-5982-0415}; {\tt hankin.robin@gmail.com}}

\begin{abstract}
The points scoring system of Formula 1 motor racing is a long-standing
contention among fans and competitors.  The inaugural points system
was $(8,6,4,3,2)$, that is, 8 points to the winner, 6 to second place,
and so on; in 2020 the points system was changed to
$(25,18,15,12,10,8,6,4,2,1)$.  However, it is difficult to assess
statistical nulls using accumulated points.  Here, I use Plackett-Luce
likelihood to identify a ranking for the competitors from the
perspective of tournament theory, this method being amenable to
statistical testing.  I go on to assess a number of reasonable points
systems objectively: in one well-defined statistical sense a Borda
points system is optimal.
\end{abstract}
\begin{keyword}
Reified Bradley-Terry\sep Likelihood\sep Formula 1 motor racing\sep Points systems
\end{keyword}


\end{frontmatter}
\linenumbers

\newcommand{\raik}{R\"{a}ikk\"{o}nen}

\section{Introduction}\label{introduction}
Formula 1 motor racing is an important and prestigious motor sport
\citep{codling2017,jenkins2010}.  Season ranking is based on a points
allocation system wherein competitors are awarded points based on race
finishing order; points accumulate additively.  The overall
competition winner is the competitor who accumulates the most points
after the final race.  The intent of the points system is to
incentivize competitors, stimulate innovation, and to create an
exciting sporting spectacle: as such, its study is a practical
application of tournament theory~\citep{lazear1981}.

However, in the case of Formula 1 motor racing, the points system is
the subject of much controversy, having changed often since the
competition's inauguration in 1950 when the points allocation was
$(8,6,4,3,2)$---eight points to the winner, six for second place, and
so on.  This system credits only the first five finishers.  As of
2020, the current points system of $(25,18,15,12,10,8,6,4,2,1)$
credits the first 10 (we ignore the bonus point awarded for fastest
lap and assume a strictly monotonic decrease).  Arguably these two
systems could introduce different rational behaviour under zero-sum
assumptions: if, in a race, a driver knows he will place seventh under
a low-risk strategy but may place sixth by dint of driving more
aggressively, the low-risk strategy might be rational under the first
points system (which does not reward the extra ranking), but not under
the second, which does.

Still, drivers have strong incentives to maximize their ranking
irrespective of any points that may be awarded: sponsors and teams
note details of drivers' performance and a great deal of personal
pride may be at stake~\citep{gayrees2019}.  It is therefore reasonable
to assume that each driver strives to maximise his rank and this will
be done here.  If this is so, then changing the points system might
change the competitors' rankings~\citep{wood2020} but not their
behaviour: surely a defect of using points to rank the competitors.

Given that racing is a zero-sum game---and that points are
monotonically decreasing---each player will try to get as high a rank
as possible regardless of the actual points system used.  However,
there are other consistent interpretations. \cite{bakhrankova2011},
for example, considers the possibility of inter-driver collusion, a
phenomenon not pursued here; and \cite{mastromarco2009} suggest that
the frequency of rule changes is driven by factors such as driver
safety and revenue optimization.

Points systems similar to that of Formula 1 are common in other racing
sports; all have the common feature of translating ranks into points
which combine additively to generate an overall ranking.  Further, we
see points systems used in the wider context of competitive situations
such as the Eurovision Song Contest and many other such group
tournaments.

\subsection{Bradley-Terry and generalizations for rank
statistics}\label{bradley-terry-and-generalizations-for-rank-statistics}

The Bradley-Terry model \citep{bradley1952} assigns non-negative
strengths $p_1,\ldots, p_n$ to each of $n$ competitors in such a way
that the probability of $i$ beating $j\neq i$ in pairwise competition
is $\frac{p_i}{p_i+p_j}$; it is conventional to normalize so that
$\sum p_i=1$.  Further, we use a generalization due to
\citet{luce1959}, in which the probability of competitor~$i$ winning
in a field of $\left\lbrace 1,\ldots, n\right\rbrace$ is
$\frac{p_i}{p_1+\cdots +p_n}$.  Noting that there is information in the
whole of the finishing order, and not just the first across the line,
we can follow \cite{plackett1975} and consider the runner-up to be the
winner among the remaining competitors, and so on down the finishing
order. Without loss of generality, if the order of finishing were
$1,2,3,4,5$, then a suitable
\citeauthor{plackett1975}-\citeauthor{luce1959} likelihood function
would be

\begin{equation}\label{competitors_1_to_5_likelihood}
\frac{p_1}{p_1+p_2+p_3+p_4+p_5}\cdot
\frac{p_2}{p_2+p_3+p_4+p_5}\cdot
\frac{p_3}{p_3+p_4+p_5}\cdot
\frac{p_4}{p_4+p_5}\cdot
\frac{p_5}{p_5}
\end{equation}

and this would be a forward ranking Plackett-Luce model in the
terminology of \cite{mollica2014}. A slight generalization allows the
incorporation of non-finishers (DNF etc). If, say, competitors 4 and 5
did not finish, we would have

\begin{equation}\label{competitors_1_to_3_only_finished}
\frac{p_1}{p_1+p_2+p_3+p_4+p_5}\cdot
\frac{p_2}{p_2+p_3+p_4+p_5}\cdot
\frac{p_3}{p_3+p_4+p_5}
\end{equation}

(observe how this likelihood function, while informative about
$p_4+p_5$, is uninformative about $p_4\left|p_4+p_5\right.$). We now
use a technique due to \cite{hankin2010,hankin2020} and introduce
fictional (reified) entities whose nonzero Bradley-Terry strength
helps certain competitors or sets of competitors under certain
conditions. The canonical example would be the home-ground advantage
in association football.  If players (teams) $1,2$ with strengths
$p_1,p_2$ compete, and if our observation were $a$ home wins and $b$
away wins for team $1$, and $c$ home wins and $d$ away wins for
team~$2$, then a suitable likelihood function would be

\[
\left(\frac{p_1+p_H}{p_1+p_2+p_H}\right)^a
\left(\frac{p_1}{p_1+p_2+p_H}\right)^b
\left(\frac{p_2+p_H}{p_1+p_2+p_H}\right)^c
\left(\frac{p_2+p_H}{p_1+p_2+p_H}\right)^d,
\]

\noindent where $p_H$ is a quantification of the beneficial home
ground effect.  Similar techniques have been used to account for the
first-move advantage in chess, and effective coordination between
members of doubles tennis teams; we may use a similar device to
account for (e.g.) wet conditions in Formula 1.  Here I analyse
seasons 2016-2019 using the \texttt{hyper2} package \citep{hankin2017}
which implements the Plackett-Luce likelihood function with additional
reified entities~\cite{hankin2020}.

One component of Formula 1 motor racing is the starting grid.  Placing
on the starting grid is determined by time trials usually driven the
day before the race itself.  Pole position is awarded to the driver
with the fastest qualifying time, and this confers a considerable
advantage to the sitter.  In this analysis we do not consider pole
position specifically, but attempt to make inferences about the time
trials and the race itself in combination (alternatively, we treat
grid placing and P-L strengths to be conditionally independent, given
race ranking).  Similarly, we treat the driver and the team as a
single entity about which we wish to make inferences.


\section{Formula 1 dataset}\label{formula-1-dataset}

\begin{table}
\centering
%\begin{tabular}{ |c|c|c|c|c|c|  ccc |c| c|c|}
\begin{tabular}{ lccccc  ccc c cc}
 \hline
          &AUS&CHN&BHR&RUS&ESP&MON&$\ldots$&BRA&ABU\\ \hline
Hamilton  &  2&  1&  2&  4&  1&  7&$\ldots$&  4&  2\\
Vettel    &  1&  2&  1&  2&  2&  1&$\ldots$&  1&  3\\
Bottas    &  3&  6&  3&  1&Ret&  4&$\ldots$&  2&  1\\
\raik &  4&  5&  4&  3&Ret&  2&$\ldots$&  3&  4\\
Ricciardo &Ret&  4&  5&Ret&  3&  3&$\ldots$&  6&Ret\\
$\ldots$&$\ldots$&$\ldots$&$\ldots$&$\ldots$&$\ldots$&$\ldots$&$\ldots$&$\ldots$&$\ldots$\\
Hartley   &  0&  0&  0&  0&  0&  0&$\ldots$&Ret& 15\\
Button    &  0&  0&  0&  0&  0&Ret&$\ldots$&  0&  0\\
Resta     &  0&  0&  0&  0&  0&  0&$\ldots$&  0&  0\\ \hline
\end{tabular}
\caption{\doublespacing 2017 Season results\label{results2017} table.
  Each row is a driver and each column (after the first) a venue. We
  see that Hamilton, the first row, came second in Australia, first in
  China, second in Bahrain, fourth in Russia, and so on (Hartley,
  Button, and Resta placed last).  In the first column, we see the
  result from Australia (AUS) in which Hamilton came second, Vettel
  first, Bottas third, and so on.  Here, ``Ret'' means retired and a
  zero entry means ``did not finish''}
\end{table}

Taking 2017 as an example, Table~\ref{results2017} shows the drivers'
ranks.  It is straightforward to translate this table into a
Plackett-Luce likelihood function using the \texttt{hyper2} package;
for simplicity we will consider only the 11 top-ranked drivers (in the
Plackett-Luce likelihood function, the performance of lower-ranked
players can be weakly informative about higher-ranked players'
strengths.  For example, we see that Vettel retired twice---in
Singapore and Japan---so any player who placed in those venues will
effectively ``steal'' strength from Vettel, and generally ``give'' it
to Hamilton or Bottas).  Although it has many terms, the overall
likelihood expression for the 2017 season is of the general form

\newcommand{\pham}{p_\mathrm{Ham}}
\newcommand{\pvet}{p_\mathrm{Vet}}
\newcommand{\pbot}{p_\mathrm{Bot}}
\newcommand{\prai}{p_\mathrm{Rai}}
\newcommand{\pric}{p_\mathrm{Ric}}
\newcommand{\pver}{p_\mathrm{Ver}}
\newcommand{\pper}{p_\mathrm{Per}}
\newcommand{\poco}{p_\mathrm{Oco}}
\newcommand{\psai}{p_\mathrm{Sai}}
\newcommand{\phul}{p_\mathrm{H\"{u}l}}
\newcommand{\pmas}{p_\mathrm{Mas}}

\begin{equation}
\frac{p_\mathrm{Ham}^{20}\,p_\mathrm{Mas}^{16}\,p_\mathrm{Bot}^{19}\ldots
}{\parbox{4in}{$(p_\mathrm{Per}+p_\mathrm{Oco})(p_\mathrm{Ver}+p_\mathrm{Per}+p_\mathrm{Sai}+p_\mathrm{Hul}+p_\mathrm{Mas})\\
    \rule{10mm}{0mm}(p_\mathrm{Ric}+p_\mathrm{Per}+p_\mathrm{Mas})(p_\mathrm{Sai}+p_\mathrm{Mas})\\ \rule{20mm}{0mm}
    (p_\mathrm{Ver}+ p_\mathrm{Per}+ p_\mathrm{Oco}+ p_\mathrm{Sai}+ p_\mathrm{Hul})\ldots
     $}}.
 \end{equation}

\begin{figure}
{\centering \includegraphics{printapiechart-1}}
\caption[\doublespacing Maximum likelihood estimates \label{piechartstrength} of the
  strengths of the top-ranked 11 drivers in Formula 1 motor racing,
  seasons 2016-2019]{\doublespacing Maximum likelihood
  estimates \label{piechartstrength} of the strengths of the
  top-ranked 11 drivers in Formula 1 motor racing, seasons
  2016-2019}\label{fig:printapiechart}
\end{figure}

Finding the maximum likelihood estimate for the players' strengths is
straightforward numerically. The \texttt{hyper2} package includes a
suite of numerical optimization routines, and because they have access
to derivatives, convergence is rapid.  A graphical diagram of the
strengths is given in Figure~\ref{piechartstrength}.  We see that in
2016 the driver with the largest estimated strength was Rosberg at
about 30\%, and in years 2017-2019 was Hamilton at about 29\%, 37\%,
and 42\% respectively. As an illustration of the value of likelihood
methods (as opposed to points-based methods), a likelihood ratio test
[\texttt{samep.test()}, supplied with the \texttt{hyper2} package]
rejects the null that Hamilton and Vettel have the same strength in
2018 $(H_0\colon p_\mathrm{Ham}= p_\mathrm{Vet}$) with a likelihood
ratio of $e^{2.76}\simeq 15.8$, corresponding by Wilks's theorem to an
asymptotic \(p\)-value of about \(0.02\). We may also use the reified
entity concept to test the null hypothesis that Hamilton's strength
was unchanged from 2016, where Rosberg had the highest estimated
strength, to 2017-2019, where Hamilton did; we fail to reject this
null.

\subsection{Likelihood scoring vs points
scoring}\label{likelihood-scoring-vs-points-scoring}

Applying the current points system, for example, to the 2017 results
table we would rank the drivers as follows:

\begin{multline}\nonumber
\mbox{Hamilton}\succ
\mbox{Vettel}\succ
\mbox{Bottas}\succ
\mbox{\raik}\succ
\mbox{Ricciardo}\succ\\
\mbox{Verstappen}\succ
\mbox{Perez}\succ
\mbox{Ocon}\succ
\mbox{Sainz}\succ
\mbox{H\"{u}lkenberg}\succ
\mbox{Massa}
\end{multline}

\noindent (this happens to be identical to the ranking after the extra
point was awarded for fastest lap).  However, if we were to follow
\citet{zipf1949} and adopt a points system of $(1,1/2,1/3,\ldots)$ we
would have

\begin{multline}\nonumber
\mbox{Hamilton}\succ
\mbox{Vettel}\succ
\mbox{Bottas}\succ
\mbox{Ricciardo}\succ
\mbox{Verstappen}\succ\\
\mbox{\raik}\succ
\mbox{Perez}\succ
\mbox{Ocon}\succ
\mbox{Massa}\succ
\mbox{Sainz}\succ
\mbox{H\"{u}lkenberg}.
\end{multline}

\noindent Thus, these two systems agree on the first three places but
fourth is awarded to \raik\ under the current F1 system and Ricciardo
under \citeauthor{zipf1949}.  Compare the likelihood ranking:

\begin{multline}\nonumber
\mbox{Hamilton}\succ
\mbox{Vettel}\succ
\mbox{Bottas}\succ
\mbox{\raik}\succ
\mbox{Ocon}\succ
\mbox{Ricciardo}\succ\\
\mbox{Verstappen}\succ
\mbox{Perez}\succ
\mbox{Massa}\succ
\mbox{Sainz}\succ
\mbox{H\"{u}lkenberg}.
\end{multline}

\begin{figure}
{\centering \includegraphics{likevspoints-1}}
\caption{\doublespacing Ordering of
  the \label{orderingbypointsandlikelihood} top-ranked 11 drivers in
  Formula 1 seasons 2016-2019.  Horizontal axis gives official
  (points-based) order, and the vertical axis gives the likelihood
  order.  Thus, taking 2017 as an example, the points-based ordering
  would be Hamilton first, then Vettel, then \raik; while the
  likelihood ordering is (reading vertically) Hamilton, Vettel, Bottas
}\label{fig:likevspoints}
\end{figure}

\noindent So for 2017 at least, we see that the current points system
agrees with likelihood ranking for the top four places, while Zipf's
law agrees to three.  We can plot one ranking against the other, shown
in Figure~\ref{orderingbypointsandlikelihood}.  Note that the
historically correct points awarded to the drivers differs from that
calculated here.  That is for two reasons: firstly, ``fastest lap''
points are not included here, and also the truncation of the order
table to the first 11 drivers can increase the rank of a driver if
non-first-11 drivers are placed.

We thus see a comparison between two ordering systems.  Taking 2017 as
an example, drivers Hamilton and Vettel are respectively first and

second according to both ranking ranking procedures; but \raik\ and
Bottas are third and fourth, and fourth and third, according to points
and likelihood respectively. We define the \emph{degree of agreement}
between the two ranking systems as the maximal value of \(r\) such
that places \(1,2,\ldots, r\) all match.  Thus, from
Figure~\ref{orderingbypointsandlikelihood}, the degree of agreement
between likelihood ranking and points ranking for the years 2016-2019
would be 11,1,2,2 respectively.

However, the points system used is essentially arbitrary.  We could
use, for example, Zipf's law to rank the drivers: award one point to
the winner, half a point to second place, one third of a point to
third, and so on; see Figure~\ref{orderingbypointsandlikelihood_zipf}
in which we see generally poorer agreement between points-based ranks
and likelihood-based ranks, with a degree of agreement of 0,1,2,2 for
the years 2016-2019 respectively.  This might be an indication that
using a Zipf law points allocation is objectively worse than the
current points system.  There are a number of plausible points systems
that might be used:

\begin{figure}
{\centering \includegraphics{likevspointszipf-1}}
\caption[\doublespacing As for
  Figure~\ref{orderingbypointsandlikelihood} \label{orderingbypointsandlikelihood_zipf}
  but points ranking calculated by Zipf's law]{\doublespacing As for
  Figure~\ref{orderingbypointsandlikelihood} \label{orderingbypointsandlikelihood_zipf}
  but points ranking calculated by Zipf's law. Note the generally
  poorer agreement between the two ranking
  systems}\label{fig:likevspointszipf}
\end{figure}

\begin{itemize}
\item The current Formula 1 system $(25,18,15,12,10,8,6,4,2,1)$
\item The inaugural Formula 1 system $(8,6,4,3,2)$
\item Zipf's law $(1,\frac{1}{2},\frac{1}{3},\frac{1}{4},\ldots)$
\item Borda $(n,n-1,n-2,\ldots,3,2,1,0)$
\item Halving system:  $(1,\frac{1}{2},\frac{1}{4},\frac{1}{8},\ldots)$
\item A ``winner takes all'' system $(1,0,\ldots)$
\end{itemize}

We note that some of these may be generalized.  We might consider a
more general Borda-like points system $(r,r-1,r-2,\ldots,3,2,1,0)$ for
fixed integer $0<r<n$ \citep{emerson2007}; the halving system can be
generalized to a geometric distribution; and the winner takes all
system can be replaced by giving equal points to the top \(r\)
competitors $(\underbrace{1,1,\ldots ,1}_{r},0,0,\ldots)$, for some
integer~$r<n$.

\begin{table}
\centering
\begin{tabular}{ |c|c|c|c|c|}
 \hline
         &    2016 & 2017 & 2018 & 2019\\ \hline
current  &    11   & 1    & 2    & 2   \\
inaugural&     8   & 1    & 2    & 2   \\
Zipf     &     0   & 1    & 2    & 2   \\
Borda    &    11   & 4    & 6    & 2   \\
WTA      &     0   & 1    & 2    & 2   \\
exp2     &     0   & 1    & 2    & 2   \\
 \hline
\end{tabular}
\caption{\doublespacing Degree of agreement for seasons 2016-9 with
  different \label{doapoints} points systems}
\end{table}


It is straightforward to calculate the degree of agreement for the
observed rank table for years 2016-2019, shown in
Table~\ref{doapoints}.  It is clear that there is no points system
that is the best for all four years.  However, observing that both the
likelihood ranking and the points ranking are random variables in this
paradigm suggests a method whereby we can objectively assess a given
points system.  Using sampling techniques we can repeatedly generate
an order table \emph{in silico}, using estimated driver strengths from
the observed tables.  For each of, say, 1000 such synthetic tables,
calculate drivers' maximum likelihood \citeauthor{plackett1975}
strengths, and also their points awarded according to any given points
system.  We then compare rankings generated by the
\citeauthor{plackett1975} strengths and the points awarded and note
the degree of agreement between the two, as measured by the number of
rankings correctly predicted.  This furnishes an objective assessment
of the points system used.

\subsection{Numerical results}\label{numerical-results}

\begin{figure}
{\centering \includegraphics{alottaplotsmeancorrect-1}}
\caption[\doublespacing Simulated Formula 1 races, seasons
  2016-2019]{\doublespacing Simulated Formula 1 races, seasons
  2016-2019: mean number of agreeing places for each of six
  points \label{simulated_mean} systems when compared with a
  Plackett-Luce strength ordering}\label{fig:alottaplotsmeancorrect}
\end{figure}

\begin{figure}
{\centering \includegraphics{alottaplotswinnercorrect-1} }
\caption[\doublespacing Simulated Formula 1 races, seasons
  2016-2019]{\doublespacing Simulated Formula 1 races, seasons
  2016-2019: probability of identifying the Plackett-Luce winner for
  each of six points \label{simulated_winner_correct}
  systems}\label{fig:alottaplotswinnercorrect}
\end{figure}

\begin{figure}
{\centering \includegraphics{alottaplotstotalrankorder-1}}
\caption[\doublespacing Simulated Formula 1 races, seasons
  2016-2019]{\doublespacing Simulated Formula 1 races, seasons
  2016-2019: probability of complete agreement between Plackett-Luce
  ranks and ranks calculated by each of six
  points \label{simulated_complete_agreement}
  systems}\label{fig:alottaplotstotalrankorder}
\end{figure}

We now assess the six points systems using the methodology outlined
above, using 1000 \emph{in silico} trials.  For each of the six points
systems, each of the 1000 trials results in a single non-negative
integer: the degree of agreement between the Plackett-Luce ranking and
the rankings according to the points system considered.  There are
three measures that might be used to assess the distribution of degree
of agreement: (1), the mean degree of agreement \(\overline{d}\); (2),
the probability of correctly predicting the winner,
$\operatorname{Prob}(d\geq 1)$; and (3), the probability of correctly
predicting the complete order statistic $\operatorname{Prob}(D=n)$.
These summaries are shown for each of the six points systems in
Figures~\ref{simulated_mean}, \ref{simulated_winner_correct}
and~\ref{simulated_complete_agreement} respectively.

We see that the Borda points system $(n,n-1,n-2,\ldots,3,2,1)$ gives
the highest mean number of places, and also the highest probability of
correctly predicting the winner.  However, the probability of predicting
the complete order statistic is maximized using the winnner takes all
system \((1,0,0,\ldots)\).

\section{Conclusions}

Many competitive situations involve ranking the participants and one
way of doing this is to assign points for the winner, second place,
third place, etc.  The overall ranking for a sequence of observations
is decided on the basis of accumulated points.  Because changing the
points system can change participants' overall ranking but does not
affect their behaviour, accumulated points should not be used to make
inferences about participants' skills.  Maximum likelihood estimation
of Plackett-Luce strengths furnishes a ranking system that does not
suffer from the arbitrariness of a points system.  Further, this
allows one to conduct statistical tests on a range of interesting
nulls in the context of an established suite of software.

The points system used in Formula 1 motor racing is a source of lively
debate from many perspectives, with changes being controversial.  By
treating total points scored as a random variable, it is possible to
compare different point allocation schemes against objective
Plackett-Luce ranks.  Of the six points systems considered here, a
Borda system or a winner-takes-all-system appear to be closest to
objective Plackett-Luce, depending on the exact definition of
``closest''.

%% BibTeX support
\bibliography{formula1}

\end{document}
