\documentclass[12pt]{article}
\usepackage{xcolor}
\begin{document}

\section*{MS596 ``Analysis of competitive surfing tournaments
  with generalized Bradley-Terry likelihoods" by Driver and
  Hankin: second rebuttal}

Below, the reviewer's comments are in black, and our replies to the
issues are in \textcolor{blue}{blue}.  We have indicated changes to
the manuscript where appropriate.  

Short story: we have accommodated all the comments with rewording and
clarification.  We have included an extended discussion of some of the
points below.  The resulting document is, we believe, stronger and
more scholarly than before and we recommend it to you.


\section*{Referee's comments}

The paper does a good job of outlining the basics of surfing in terms
of competition rules and structure, and frames up the problem at
hand.  However, the paper itself feels rather light on content.  The
authors jump almost immediately to the results section and don't go
into very much detail in terms of how or why Bradley-Terry models are
used, other than the fact that they are used.

\textcolor{blue}{We have re-read our manuscript and indeed
  Bradley-Terry (BT) does make a rather abrupt entrance.  However, we
  do describe and justify and motivate the use of BT in section three.
  On reflection we believe we can address this comment by rearranging
  sections 2 and 3: this gives a gentler and more appropriate
  introduction to BT and indeed is a more appropriate structure for a
  manuscript.  In line with this, we have changed the section heading
  from ``previous related work'' to ``Previous statistical analysis of
  competitive surfing'' which is a better description, we feel.}

The authors are guilty
of referring to various numerical quantities throughout the paper,
without expanding on how these values were calculated, and what
implications the values have. The paper would benefit from more
explicit equations throughout; on several occasions the authors refer
to modifying or changing their likelihood function to suit their
problem at hand, but it is difficult for the reader (or at least
myself!) to intuitively understand what this modified function would
look like, without writing it down or thinking hard about
it. Throughout the paper I found I had to do a lot of the mathemati
cal and abstract thinking myself, which would have been made much
easier with more equations and explicit definitions and derivations of
parameters, likelihood functions, and important quantities that are
referred to in Sections 5, 6, and 7.

The figures and tables are poorly presented and would benefit from
significant rework in terms of formatting.

Section 7, while an interesting topic, feels like it falls more into
the realm of speculation than concrete analysis in the way it is
presented. The methodology is not very clear and it would be nice to
see a more explicit definition and derivation of pc

Overall, it feels like the main conclusion is "not all surfers have
the same skill". This is not insignificant, however, it is not
particularly surprising and is one that could have been reached using
a number of different methods. Had the authors done a more complete
job of fleshing out why the Bradley-Terry approach was better for
reaching this conclusion than others, I would be more satisfied, but
as it stands it falls a little flat.

From a style perspective, the paper is well-written and enjoyable to
read, which makes the lack of substantial content a bit of a shame, as
the authors are clearly able to write a paper worth reading. Overall,
it sounds like the authors do have the data and analysis available to
warrant a more complete paper, however, what they have submitted feels
rushed and rather rough around the edges. While possible, it would
take a fairly significant overhaul/expansion to the current version of
the paper to get it into a publication-ready format.

\end{document}
