\documentclass[12pt]{article}
\usepackage{xcolor}
\begin{document}

\section*{MS596 ``Analysis of competitive surfing tournaments with generalized Bradley-Terry likelihoods": response to minor comments}

Below, we reproduce the reviewers' minor comments; our replies to
the issues are in \textcolor{blue}{blue}.  We have indicated changes
to the manuscript where appropriate.

Short story: each of the  comments is perfectly reasonable and we have 
changed the manuscript as suggested.




Overview I am pleased to see the authors have addressed the issues
identified in the first review of the paper, or in one or two cases,
have provided reasonable justification for their approach. The changes
to the general structure have improved the paper’s overall readability
and flow, and I commend the inclusion of more explicit derivations and
references to numerical quantities. As such, I am satisfied with the
revisions the authors have made and I am happy to recommend the paper
be accepted, conditional on some minor revisions, noted below:



Contextual changes and clarifications

Page 2: The first reference to the ``World Championship Tour'' would
benefit from including a one-off abbreviation (WCT), which is referred
to throughout the data set.

\textcolor{blue}{Done; WSL used throughout for ``World Surf League''
  and minor rewording ensures that the phrase ``World Championship
  Tour'' does not appear}

Page 4: It is mentioned in the abstract that the analysis is performed
on data from the 2019 WSL, however a reference in the main text at the
top of Section 5 would help clarify this.

\textcolor{blue}{Done}


Page 4: The authors mention the surfer Vries---who does not appear to
be a wildcard---whose maximum likelihood strength is zero. Why is
this? My guess is that Vries has never won a competitive surfing heat
(and therefore there is no information on his true, underlying
ability), however a brief clarification here woud be useful.

\textcolor{blue}{The referee is quite right: Vries competed only twice
  in 2019, coming last (second and third) in each of his two sessions.
  Brief explanation added to he mauscript.}


Page 6 / Figure 1: I'm pleased to see the authors replace the old pie
chart in Figure 1 with a dot chart---this is a much clearer
visualization of the results.  On Page 6 the authors state: ``Figure 1
shows the maximum likelihood estimate for the 23 competitors'
strengths, which appears to show a wide range from Florence at about
$17.5\%$ down to Morais at about $1\%$". What exactly do the
quantities $17.5\%$ and $1\%$ refer to/how are they computed? A quick
clarification of these quantities would help explain the practical
implications of the findings.

\textcolor{blue}{The figures are, as stated, the maximum likelihood
  estimates of the competitors' strengths.  We think that our use of
  percentages (which add to 100) instead of strengths (which add to 1)
  was the cause of the confusion and we have amended the manuscript to
  use strengths, not percentages.}

Page 7/Figure 2: The authors suggest there is a loose positive
correlation between maximum likelihood strength and total points,
which is undoubtedly true. However, the x-axis on Figure 2 itself has
been distorted, which makes the trend appear more linear than it
really is. I'd prefer to see the x-axis undistorted, unless the
authors are explicitly trying to make a point about the trend being
log-linear, in which case I'd like to see this briefly discussed.

\textcolor{blue}{Horizontal axis now linear; no other change except
  for tiny movement of labels so they do not obscure one another.  We
  are mortified to see that the linearity is more apparent without the
  log-transform.}

Page 7: The authors suggest we may reject the hypothesis that the
strength of the entity exceeds about 0.025. Does this assertation come
with a supporting p-value? It is unclear from Figure 4 how this
conclusion is reached.

\textcolor{blue}{This was a (devil-inspired) typo.  The value should
  read 0.015.  The reasoning was that at 0.015 the support was -2.  We
  have ddded a sentence of explanation on p7 and also the figure
  caption.}


Grammatical changes

Page 3: Change ``Further, we use a generalization due to (Luce,
1959)..." to ``Further, we use a generalization due to Luce
(1959)...".

Page 7: Change ``..., but in addition we may reject the
the proposal that..." to ``..., but in addition we may reject the
proposal that...".

\textcolor{blue}{Done}

Page 10: References with URL's should be split over multiple lines to avoid being cut-off the page.

\textcolor{blue}{Done}

Page 11: Table 1 could be formatted to fit within the page margins.

\textcolor{blue}{Done}

Page 17: Capitalisation of Figure 2 axis titles.

\textcolor{blue}{Done}

Page 17: Figure 2 caption is a little brief and could better explain
what is observed in the plot (e.g. the positive correlation between
variables). Especially as the authors have distorted the x-axis and
fitted a linear trend through the data. What is the reason for this?
As mentioned in a previous comment, I’d prefer to see the x-axis in
its raw form, unless there is a reason it has been distorted.

\textcolor{blue}{More detail added in caption}



Page 18: Figure 3 title – this doesn’t provide the reader with any
additional information. It does seem a little odd that Figure 2 does
not have a title, Figure 3 does, and Figure 4 does not. For
consistency, the authors could simply remove the title from Figure 33,
as it does not really add any value.

\textcolor{blue}{Done (title removed from figure 3 as suggested)}

Page 19: Figure 4 Capitalisation of y-axis title.  General comment:
should the Table and Figure captions have a full-stop at the end of
the last sentence?

General comment
As part of a potential future publication in the field, I’d be curious to see the methodology in Section
7 applied to World Surf League data from events that occurred prior to the announcement that surfing
would be in the Olympics. Comparing these results with those in the paper that use data from after
this announcement (perhaps supplemented with additional data and results from post-2019), would
help in our understanding whether there truly has been an observable change in non-competitive
behaviour between compatriots. Any such analysis is clearly well beyond the realm of minor revisions
and is not something I expect of the authors prior to final submission; however, I thought it could be
an interesting next step for the application of the methodology outlined in the paper!




\end{document}


