\documentclass[12pt]{article}
\usepackage{xcolor}
\begin{document}

\section*{MS596 ``Analysis of competitive surfing tournaments with generalized Bradley-Terry likelihoods": response to minor comments}

Below, we reproduce the reviewers' minor comments; our replies to
the issues are in \textcolor{blue}{blue}.  We have indicated changes
to the manuscript where appropriate.

Short story: each of the  comments is perfectly reasonable and we have 
changed the manuscript as suggested.




Overview I am pleased to see the authors have addressed the issues
identified in the first review of the paper, or in one or two cases,
have provided reasonable justification for their approach. The changes
to the general structure have improved the paper’s overall readability
and flow, and I commend the inclusion of more explicit derivations and
references to numerical quantities. As such, I am satisfied with the
revisions the authors have made and I am happy to recommend the paper
be accepted, conditional on some minor revisions, noted below:



Contextual changes and clarifications

Page 2: The first reference to the ``World Championship Tour'' would
benefit from including a one-off abbreviation (WCT), which is referred
to throughout the data set.

\textcolor{blue}{Done; WSL used throughout for ``World Surf League''
  and minor rewording ensures that the phrase ``World Championship
  Tour'' does not appear}

Page 4: It is mentioned in the abstract that the analysis is performed
on data from the 2019 WSL, however a reference in the main text at the
top of Section 5 would help clarify this.

\textcolor{blue}{Done}


Page 4: The authors mention the surfer Vries---who does not appear to
be a wildcard---whose maximum likelihood strength is zero. Why is
this? My guess is that Vries has never won a competitive surfing heat
(and therefore there is no information on his true, underlying
ability), however a brief clarification here woud be useful.

\textcolor{blue}{The referee is quite right: Vries competed only twice
  in 2019, coming last (second and third) in each of his two sessions.
  Brief explanation added to he mauscript.}


Page 6 / Figure 1: I'm pleased to see the authors replace the old pie
chart in Figure 1 with a dot chart---this is a much clearer
visualization of the results.  On Page 6 the authors state: ``Figure 1
shows the maximum likelihood estimate for the 23 competitors'
strengths, which appears to show a wide range from Florence at about
$17.5\%$ down to Morais at about $1\%$". What exactly do the
quantities $17.5\%$ and $1\%$ refer to/how are they computed? A quick
clarification of these quantities would help explain the practical
implications of the findings.

\textcolor{blue}{The figures are, as stated, the maximum likelihood
  estimates of the competitors' strengths.  We think that our use of
  percentages (which add to 100) instead of strengths (which add to 1)
  was the cause of the confusion and we have amended the manuscript to
  use strengths, not percentages.}

Page 7/Figure 2: The authors suggest there is a loose positive correlation between maximum likelihood
strength and total points, which is undoubtedly true. However, the x-axis on Figure 2 itself has been
distorted, which makes the trend appear more linear than it really is. I’d prefer to see the x-axis
undistorted, unless the authors are explicitly trying to make a point about the trend being log-linear,
in which case I’d like to see this briefly discussed.
Page 7: The authors suggest we may reject the hypothesis that the strength of the entity exceeds
about 0.025. Does this assertation come with a supporting p-value? It is unclear from Figure 4 how
this conclusion is reached.

Grammatical changes
Page 3: Change “Further, we use a generalization due to (Luce, 1959)...” to “Further, we use a
generalization due to Luce (1959)...”.
Page 7: Change “..., but in addition we may reject the the proposal that...” to “..., but in addition we
may reject the proposal that...”.

\end{document}


