\documentclass[review]{elsarticle}

\usepackage{lineno,hyperref}
\usepackage{amsmath}
\usepackage{amssymb}
\usepackage{booktabs}
\usepackage{units}
\usepackage{xcolor}
\modulolinenumbers[5]

\journal{Journal of Economic Behavior and Organization}

%%%%%%%%%%%%%%%%%%%%%%%
%% Elsevier bibliography styles
%%%%%%%%%%%%%%%%%%%%%%%
%% To change the style, put a % in front of the second line of the current style and
%% remove the % from the second line of the style you would like to use.
%%%%%%%%%%%%%%%%%%%%%%%

%% Numbered
%\bibliographystyle{model1-num-names}

%% Numbered without titles
%\bibliographystyle{model1a-num-names}

%% Harvard
\bibliographystyle{model2-names.bst}\biboptions{authoryear}

%% Vancouver numbered
%\usepackage{numcompress}\bibliographystyle{model3-num-names}

%% Vancouver name/year
%\usepackage{numcompress}\bibliographystyle{model4-names}\biboptions{authoryear}

%% APA style
%\bibliographystyle{model5-names}\biboptions{authoryear}

%% AMA style
%\usepackage{numcompress}\bibliographystyle{model6-num-names}

%% `Elsevier LaTeX' style
%\bibliographystyle{elsarticle-num}
%%%%%%%%%%%%%%%%%%%%%%%


\begin{document}

\begin{frontmatter}

\title{A generalization of the Bradley-Terry model for draws in chess}

%% Group authors per affiliation:
\author{Robin K. S. Hankin\fnref{myfootnote}}
\address{AUT University, 2-14 Wakefield Street, Auckland, New Zealand}



\begin{abstract}
\definecolor{DarkGreen}{RGB}{14,159,4}
In inference problems where the dataset comprises Bernoulli outcomes
of paired comparisons, the Bradley-Terry model offers a simple and
easily interpreted framework.  However, it does not deal easily with
chess because of the existence of draws, and the white player
advantage.  Here I present a new generalization of Bradley-Terry in
which a chess game is regarded as a three-way competition between the
two players and an entity that wins if the game is drawn.
Bradley-Terry is then further generalized to account for the white
player advantage by positing a second entity whose strength is added
to that of the white player.  These techniques afford insight into
players' strengths, response to playing black or white, and
risk-aversion as manifested by probability of drawing.  The likelihood
functions arising are easily optimised numerically.  I analyse a
number of datasets of chess results, including the infamous 1963 World
Chess Championships, in which Fischer accused three Soviet players of
collusion.  I conclude, on the basis of a dataset that includes only
the scorelines at the event itself, that the Candidates Tournament
(Cura\c{c}ao 1962) exhibits evidence of collusion ($p<10^{-5}$), in
agreement with previous work.  \textcolor{DarkGreen}{I also present
  scoreline evidence for the effectiveness of such a drawing cartel:
  noncollusive games are less detrimental to future play than
  collusive games ($p<10^{-6}$).}
\end{abstract}

\begin{keyword}
  Chess draws\sep Bradley-Terry\sep Likelihood \sep Chess Collusion
\end{keyword}

\end{frontmatter}

\linenumbers

\newcommand\headercell[1]{%
   \smash[b]{\begin{tabular}[t]{@{}c@{}} #1 \end{tabular}}}
\newcommand{\draw}{\ensuremath{\mathfrak{D}}}
\newcommand{\cdraw}{\ensuremath{\mathfrak{C}}}
\newcommand{\white}{\ensuremath{\mathfrak{W}}}
\newcommand{\Prob}{\operatorname{Prob}}
\definecolor{DarkGreen}{RGB}{14,159,4}

\section{Introduction: the Bradley-Terry model}

Many situations involve pairwise comparisons that amount to a
Bernoulli trial.  The canonical application would be two-way
competitive situations such as tennis in which one player is
identified as the winner.  The Bradley-Terry
model~\citep{bradley1952,turner2012} considers~$n$ players and assigns
a non-negative number $p_i$ to player $i$, with $\sum p_i=1$.  Then
player $i$ beats player $j$ with probability $\frac{p_i}{p_i+p_j}$.
This simple system has been modified and extended many times, for
example to account for multi-way competitions such as {\em MasterChef
  Australia}~\citep{hankin2017}, and non-competitive situations such
as consumer behaviour~\citep{hankin2010} and mating
preference~\citep{west2008}.

Bradley-Terry is not directly applicable to chess for two reasons:
firstly, the white player appears to be at a considerable advantage,
winning about 37.5\% of tournament games (black wins 27.6\%); and
secondly, many chess games, about 39.4\%, are drawn (figures from
Chessbase).  \textcolor{orange}{First-mover advantage is closely
  related to incumbent advantage, and the techniques presented here
  have been applied to ecological incumbency~\cite{hankin2010}.  More
  generally, \cite{szymanski2003} considers sporting contests as
  examples of competitive situations in which a contribution of some
  kind [effort, investment, attendance of the elite, etc] is elicited
  from contestants who may as a result win a prize.
  \citeauthor{szymanski2003} points out that this paradigm arguably
  covers executive corporate promotion, interviews for a position in
  an office, and most sporting tournaments.  As such, the phenomenon
  of collusion and indeed drawing in pairwise comparison is surely an
  interesting phenomenon.}
  
\subsection{Draws} Chess games may be won by either player, or may be
drawn; one inference of a draw is that neither player is better than
the other.  Before 1867, tournament games that were drawn were
replayed until one or other player won---reducing a chess game to a
Bernoulli trial (with probability one).  This system became unworkable
due to the large number of draws, resulting in the modern practice of
awarding each player of a drawn game half a point, keeping chess a
zero-sum game.  Drawn games are generally considered to be undesirable
as they are usually poorer in spectacular or striking elements such as
sharp tactical combinations or sacrifice.  Excessive drawing dissuades
sponsors, leaves spectators feeling cheated, and gives commentators
less material to discuss.  Players themselves do not seem to value
draws in the wider context of the game: \citet{fischer1969},
for example, includes only 8 draws in his compilation of 60 memorable
games, surely not representative of tournament play in which about
39.4\% of games are drawn (binomial test, $p<10^{-5}$).

\citet{moul2009} conceptualise a player as being able to adopt a
strategy on a spectrum ranging from conservative to aggressive, thus
being able to exert some control over the probability of a draw and
suggest that risk-averse players may pursue less vigorous strategies,
trading expected payout for reduced variance.  Further,
\citet{langer2013} point out that rational players in an ongoing
competitive situation modify their strategies depending on whether
they are leading or lagging: the lagging player will adopt a higher
risk strategy (`nothing to lose'), while the leading player is
incentivised to protect the lead by lowering the risk of losing.  In a
chess context, this means the leading player will be more likely to
aim for a draw.

Such arguments suggest that one should infer more from a drawn game
than simple equality of skill: the play itself tends to be different
and more cautious, the drawing players have a different mindset, and
drawn games have less value in the wider context of chess.

\subsection{First-mover advantage} Traditionally White moves first in
chess, and this feature introduces an asymmetry into play.
Historically, we see about 1.35 white wins for every black win.
However, interpretations of this statistic vary and the issue is hotly
debated with opinions ranging from that of \citet{adams2007}---``White
to play and win'' to \citet{adorjan1988}---``Black is OK!''.

\subsection{Generalized Bradley-Terry}

Here I show how Bradley-Terry may be generalized to account for the
existence of draws and the first-mover advantage.  Consider first the
standard Bradley-Terry model for a finite number of chess players.
Ignoring the effect of playing white, we wish to find numbers
(``strengths") $p_i\geqslant 0$ for $1\leqslant i\leqslant n$ with
$\sum p_i=1$ such that the probability of $i$ beating $j\neq i$ is
$\frac{p_i}{p_i+p_j}$.  A suitable likelihood function, on the
assumption of independence of trials, would be

\begin{equation}
\prod_{i<j}
\left(\frac{p_i}{p_i+p_j}\right)^{W_{ij}}
\left(\frac{p_j}{p_i+p_j}\right)^{L_{ij}}
\end{equation}

\noindent where $W_{ij}$ is the number of times $i$ beats $j$ and $L_{ij}$ is
the number of times $i$ loses to $j$.  Here I present a new
generalization of Bradley-Terry in which a chess game is regarded as a
three-way competition between the two players and a third entity (a
``draw monster'') that wins if the game is drawn.  The system may be
further modified to account for first-move advantage by positing an
entity whose strength is added to that of the white player.  This
methodology affords insight into players' strengths, response to
playing black or white, and risk-aversion as manifested by probability
of drawing.  If we consider a chess game to be a three-way trial, we
seek a likelihood function for non-negative $p_1,\ldots,p_n,\draw$
(here \draw\ is the draw monster's strength, fraktur font being used
for reified entities' strengths) with $\draw+\sum p_i=1$.  A suitable
likelihood function would be

\begin{equation}
\prod_{i<j}
\left(\frac{p_i}{p_i+p_j+\draw}\right)^{W_{ij}}
\left(\frac{p_j}{p_i+p_j+\draw}\right)^{L_{ij}}
\left(\frac{\draw  }{p_i+p_j+\draw}\right)^{D_{ij}}
\end{equation}

\noindent where $D_{ij}$ is games drawn.  However, this approach
ignores the extra strength conferred by the first-move advantage.  We
thus consider another entity whose strength \white\ is added to that
of the white player.  The strength of the white player is thus the sum
of the player's strength and \white.  Recalling that a game might be a
draw, we obtain the probabilities as given in
Table~\ref{probabilities}.  The corresponding likelihood function is
defined on $\white+\draw+\sum p_i=1$, given below in equation
\ref{likelihood_WD}:

\newcommand{\oo}{\vphantom{W^{W^{W^W}}}}

\begin{table}
  \caption{Probabilities \label{probabilities}of a win, draw, loss for the two players under a generalized Bradley-Terry model}
\centering
\fbox{%
\begin{tabular}{l|cc}\\
  &$i$ plays white & $j$ plays white\\ \hline
$\Prob{(\mbox{$i$ wins})}$  & $\frac{p_i+\white\oo}{p_i+p_j+\white+\draw}$ & $\frac{p_i}{p_i+p_j+\white+\draw}$  \\
$\Prob{(\mbox{draw})}$      & $\frac{\draw\oo}{p_i+p_j+\white+\draw}$     & $\frac{\draw}{p_i+p_j+\white+\draw}$    \\
$\Prob{(\mbox{$i$ loses})}$ & $\frac{p_j\oo}{p_i+p_j+\white+\draw}$   & $\frac{p_j+\white}{p_i+p_j+\white+\draw}$\\
\end{tabular}}
\end{table}

\begin{align}\nonumber
\prod_{i<j}&{}
\left(\frac{p_i+\white}{p_i+p_j+\draw+\white}  \right)^{W_{ij}^\mathrm{white}}
\left(\frac{p_j}{p_i+p_j+\draw+\white    }\right)^{L_{ij}^\mathrm{white}}\\
&{} \left(\frac{p_i}{p_i+p_j+\draw+\white}\right)^{W_{ij}^\mathrm{black}}\label{likelihood_WD}
\left(\frac{p_j+\white}{p_i+p_j+\draw+\white  }\right)^{L_{ij}^\mathrm{black}}\\
&{} \left(\frac{\draw }{p_i+p_j+\draw+\white  }\right)^{D_{ij}^\mathrm{white}+D_{ij}^\mathrm{black}}
\nonumber
\end{align}

\noindent where $W_{ij}^\mathrm{white}$ is the number of games won by
$i$ when playing white against $j$, and so on.  Note that the power of
the final term, $D_{ij}^\mathrm{white}+D_{ij}^\mathrm{black}$ is the
total number of games drawn between $i$ and $j$ irrespective of
colour.  We now apply this and related likelihood models to real chess
datasets using the Method of Support~\citep{edwards1972}.  To test any
hypothesis $H$ on likelihood function $\mathcal{L}$ one first
identifies the constraint on hypothesis space corresponding to $H$.
One then finds the unconstrained maximum likelihood
$\mathcal{L}_\mathrm{free}$ and compares it with the maximum
likelihood $\mathcal{L}_\mathrm{cons}$ when the optimization is
constrained by $H$.  The test statistic is then the likelihood ratio
$\mathcal{L}_\mathrm{free}/\mathcal{L}_\mathrm{cons}$ or, equivalently
the support $\mathcal{S}=\log\mathcal{R}$.  Wilks's theorem states
that the asymptotic distribution of $2\mathcal{S}$ is $\chi^2_n$ where
$n$ is the number of degrees of freedom implemented by $H$.

The above likelihood functions are objective and reasonable
representations of chess game outcomes.  However, they differ markedly
from previous likelihood functions of draws in that they include
reified entities representing proclivity to draw \draw, and white or
first-mover advantage \white.

\subsection{Previous work}

\citet{davidson1970} uses a threshhold approach to interpret draws.
From a chess draw, he infers that the performance of the players is
identical, within a threshhold.  Using a logistic expression for the
difference in players' performance he uses, in current notation,
$\draw=\nu\sqrt{p_ip_j}$.  However, \citet{joe1990} uses likelihood
methods to show that this model does not fit data well.  Here we note
that such a model automatically gives stronger players a higher draw
potential, and also note that it is difficult to account for
first-player advantage in this framework.  Such an approach is
numerically challenging in the current framework as the unit-sum
constraint becomes nonlinear, making likelihood maximization
difficult.

\textcolor{blue}{The seminal work of \citet{moul2009} presents
  sophisticated regression techniques to analyse draws and first-mover
  advantage.  They compare all-play-all events with knockout format
  events to test---and reject---hypotheses of a distinct Soviet
  style-of-play; the work I present also considers Soviet proclivity
  to draw, using data from a single event, specifically the Stockholm
  Interzonal tournament of (1962).}

\subsection{Numerical considerations}

The Method of Support \citep{edwards1972} involves only numerical
optimization (not integration which is computationally expensive).
The analysis presented here uses the framework of the
R~\citep{rcore2019} {\tt hyper2} package~\citep{hankin2017} as a
convenient and unified interface.  Because gradients are available,
convergence to the evaluate is generally rapid; hypothesis-specific
constraints on parameter space are, in general, linear and
straightforward to implement.  The total runtime for all the tests
presented here is less than a few seconds on a desktop PC.

\section{Anand, Karpov, Kasparov}

Grandmasters Anand, Karpov, and Kasparov have a long-standing rivalry
and have played many games amongst themselves.  A summary may be found
in Table~\ref{kka}, \textcolor{red}{showing games played from 1981 to
  2017.  During this time, the players' strengths might have changed
  substantially.  Statistical assessment of time-varying strength is
  possible with the reified entity technique and this is discussed
  under further work below.  However, we may treat the strengths as
  constant by adopting one of two approaches: either we assume that
  the strengths do not change, or alternatively we interpret the
  strengths as measures of ability {\em averaged over a playing
    career}.  Either way,} Equation~\ref{likelihood_WD} becomes

\begin{table}
\caption{Results of 365 chess games \label{kka} between Grandmaster
  Anand, Grandmaster Karpov, and Grandmaster Kasparov.  For example,
  Anand played white against Karpov a total of $18+20+5=43$ times and
  won 18, drew 20, lost 5}
\centering
\fbox{%
\begin{tabular}{@{} *{4}{c} @{}}
\headercell{\small{Plays White,}\\ \small{wins}} & \multicolumn{3}{c@{}}{\small{Plays black, loses}}\\
\cmidrule(l){2-4}
& An &  Kr & Ks        \\ 
\midrule
  An & -  &  18 &   6  \\
  Kr &  7 &  -  &  18  \\
  Ks & 15 &  30 &  -   \\
\end{tabular}\quad
\begin{tabular}{@{} *{4}{c} @{}}
\headercell{\small{Plays W,}\\ \small{draws}} & \multicolumn{3}{c@{}}{\small{Plays black, draws}}\\
 \cmidrule(l){2-4}
& An &  Kr & Ks        \\ 
\midrule
  An & -  &  20 &  20  \\
  Kr & 29 &  -  &  72  \\
  Ks & 26 &  57 &  -   \\
\end{tabular}\quad
\begin{tabular}{@{} *{4}{c} @{}}
\headercell{\small{Plays W,}\\ \small{loses}} & \multicolumn{3}{c@{}}{\small{Plays black, wins}}\\
\cmidrule(l){2-4}
& An &  Kr & Ks        \\ 
\midrule
  An & -  &   5 &   11  \\
  Kr & 13 &  -  &    9  \\
  Ks &  2 &   7 &  -   \\
\end{tabular}}
\end{table}

\begin{equation}\label{kka_likelihood}
%\mathcal{L}\left(p_\mathrm{An},p_\mathrm{Kr},p_\mathrm{Ks},\white,\draw\right)=
\frac{
p_\mathrm{An}^{15}
p_\mathrm{Kr}^{12} 
p_\mathrm{Ks}^{20}
\draw^{224}
\left(p_\mathrm{Kr} + \white\right)^{25}
\left(p_\mathrm{Ks} + \white\right)^{45}
\left(p_\mathrm{An} + \white\right)^{24}
}{
\left(p_\mathrm{Kr} + p_\mathrm{Ks} + \white + \draw\right)^{193}
\left(p_\mathrm{Kr} + p_\mathrm{An} + \white + \draw\right)^{92}
\left(p_\mathrm{Ks} + p_\mathrm{An} + \white + \draw\right)^{80}
}.
\end{equation}
                                    
\begin{figure}[t]
\begin{center}
\makebox{\includegraphics{collusion_chess_draws-ddc.pdf}}
\caption{Maximum likelihood estimate for the strengths of the 3 Grandmasters in Table~\ref{kka} 
\label{maxp_karpov_kasparov_anand} under a likelihood defined as Equation~\ref{likelihood_WD}}
  \end{center}
\end{figure}

This likelihood function is easily maximized numerically and the
evaluate is shown visually in Figure~\ref{maxp_karpov_kasparov_anand},
which suggests a number of interesting and testable statistical
hypotheses.  Firstly, we observe that $\draw=0$ cannot possibly be the
case: the likelihood for this hypothesis would be identically zero,
owing to the nonzero power of \draw\ in the numerator of
equation~\ref{kka_likelihood}.  From the figure, we see that the
maximum likelihood estimate for draw strength $\hat{\draw}\simeq
0.53$: compare the na\"ive estimate of $\frac{224}{365}\simeq 0.61$
given by observing 224 draws among 265 games.  The ML estimate is
lower because the maximization process automatically adjusts for the
different players' strengths.  Secondly, we may make inferences about
$\white$, the extra strength conferred by playing white: it is
conceivable that $\white=0$ as this term does not appear on its own in
the numerator of any term.  We may test $H_0\colon \white=0$ using the
Method of Support by maximizing the likelihood subject to the
constraint $\white=0$ and comparing with the likelihood when the
optimization is free.

The result of numerical optimization is about 7.01 units of support,
or a likelihood ratio of $e^{7.01}\simeq 1107.7$: that is, well over
the two-units-of-support per degree of freedom suggested by Edwards;
we can convincingly reject the hypothesis that $\white=0$ and infer
that playing white confers a significant advantage,
consistent with expectation.  It is now possible to go beyond
merely establishing the existence of the effect, and to quantify it
accounting for the differential strengths of the players.  Taking the
MLE above, and considering a game between say Karpov and Anand, the
strengths are Karpov=0.064, Anand=0.151, \white =0.104 and \draw
=0.532.  Thus we would estimate that Karpov playing white against
Kasparov would win with probability
$\frac{0.064+0.104}{0.064+0.104+0.148+0.532}\simeq 0.198$, draw with
probability $\frac{532}{64+104+148+532}\simeq 0.625$, and lose with
probability $\simeq 0.174$ (compare $+0.075\ {=}0.627\ {-}0.297$ if
Karpov plays black).

One very natural and interesting hypothesis would be that the three
players are of equal strength
%equal_strength_is_assessed_in_chunk_kka_equal_strength_above____________
and we might hypothesise that
$p_\mathrm{Kr}=p_\mathrm{Ks}=p_\mathrm{An}$.  This too is readily
testable by the device of maximizing the support, subject to the
constraint that $p_\mathrm{Kr}=p_\mathrm{Ks}=p_\mathrm{An}$.  In this
case the extra support in going from the constrained to the free
optimization is about 4.93, or a likelihood ratio of about $138$.
However, this time the constraint corresponds to a loss of {\em two}
degrees of freedom so we would need 4 units of support, which is
attained in this case.  Alternatively we might follow Wilks, and
observe that the value of $2\mathcal{S}=2*4.92=9.84$ lies in the tail
region of its asymptotic $\chi^2_2$ distribution with a $p$-value of
about 0.0072.  We can justifiably assert that the players do indeed
have different strengths, even if we account for white advantage and
the proclivity to draw.

\subsection{Each player has a draw monster}

As argued above, players with a cautious style tend to draw and this
is an important and interesting feature of the game (even if drawn
games themselves are boring).  One line of investigation might be a
likelihood function that can account for differences in draw
proclivity among the players.  This may be achieved by considering a
game between two players to be a contest between five entities: the
two real players, the white advantage, and two {\em personalised} draw
entities, one for each real player; a drawn game is counted as a win
for either of the draw monsters.  Under these assumptions,
Table~\ref{kka} translates into the following likelihood function:

%\begin{figure}[htbp]
%\begin{center}
%<<ddcc,echo=TRUE,fig=TRUE>>=
%pie(maxp(kka_3draws))
%@
%\caption{Maximum likelihood estimate for the strengths of the 3 Grandmasters in Table~\ref{kka} 
%under a likelihood function that admits differential proclivities to draw}
%  \end{center}
%\end{figure}

\begin{equation}\label{KKA_personalised_drawmonster}
%\mathcal{L}\left(
%p_\mathrm{An},p_\mathrm{Kr},p_\mathrm{Ks}
%,\white,
%\draw_\mathrm{An},\draw_\mathrm{Kr},\draw_\mathrm{Ks}
%\right)=
\frac{
\parbox{3in}{$
p_\mathrm{An}^{15}
p_\mathrm{Kr}^{12}
p_\mathrm{Ks}^{20}
\left(p_\mathrm{Kr} + \white\right)^{25}
\left(p_\mathrm{Ks} + \white\right)^{45} 
\left(p_\mathrm{An} + \white\right)^{24}\\ \rule{10mm}{0mm}
\left(\draw_\mathrm{Kr} + \draw_\mathrm{Ks}\right)^{129}
\left(\draw_\mathrm{Kr} + \draw_\mathrm{An}\right)^{49}
\left(\draw_\mathrm{Ks} + \draw_\mathrm{An}\right)^{46}
$}}{\parbox{3in}{$
\left(p_\mathrm{Kr} + p_\mathrm{Ks} + \white + \draw_\mathrm{Kr} + \draw_\mathrm{Ks}\right)^{193}\\ \rule{10mm}{0mm}
\left(p_\mathrm{Kr} + p_\mathrm{An} + \white + \draw_\mathrm{Kr} + \draw_\mathrm{An}\right)^{92}\\ \rule{20mm}{0mm}
\left(p_\mathrm{Ks} + p_\mathrm{An} + \white + \draw_\mathrm{Ks} + \draw_\mathrm{An}\right)^{80}$}}
\end{equation}


\begin{figure}[t]
\begin{center}
\makebox{\includegraphics{collusion_chess_draws-007.pdf}}
\caption{Maximum likelihood estimate for \label{kka_3draws} the support function given in equation  \ref{KKA_personalised_drawmonster}} 
  \end{center}
\end{figure}


Figure~\ref{kka_3draws} shows the maximum likelihood estimate for the
terms of equation~\ref{KKA_personalised_drawmonster}.  We see
considerable variability in proclivity to draw.  A null of equal
proclivity to draw may be tested with the Method of Support, which
gives a $p$-value of about 0.039, borderline significant: an
interesting result, given that in this case no draw monster appears
alone anywhere in the likelihood function.

\subsection{Each player has a distinct white monster}

We now consider the possibility that the strength conferred by playing
white is specific to each player and we denote the strengths
$\white_\mathrm{An}$, $\white_\mathrm{Kr}$ and $\white_\mathrm{Ks}$.
The corresponding likelihood function would be

\begin{equation}\label{KKA_personalised_whitemonster}
%\mathcal{L}\left(
%p_\mathrm{An},p_\mathrm{Kr},p_\mathrm{Ks},
%\white_\mathrm{An},\white_\mathrm{Kr},\white_\mathrm{Ks},\draw
%\right)=
\frac{
%\parbox{3in}
%       {$
p_\mathrm{An}^{15}
p_\mathrm{Kr}^{12}
p_\mathrm{Ks}^{20}
\draw^{224}
\left(p_\mathrm{An} + \white_\mathrm{An}\right)^{24}%\\ \rule{10mm}{1mm}
\left(p_\mathrm{Kr} + \white_\mathrm{Kr}\right)^{25}
\left(p_\mathrm{Ks} + \white_\mathrm{Ks}\right)^{45}
%$}
}{\parbox{4in}{$
\left(p_\mathrm{An} + p_\mathrm{Kr} + \white_\mathrm{An} +  \draw\right)^{43}
\left(p_\mathrm{An} + p_\mathrm{Kr} + \white_\mathrm{Kr} +  \draw\right)^{49}\\ \rule{10mm}{0mm}
\left(p_\mathrm{An} + p_\mathrm{Ks} + \white_\mathrm{An} +  \draw\right)^{37}
\left(p_\mathrm{An} + p_\mathrm{Ks} + \white_\mathrm{Ks} +  \draw\right)^{43}\\ \rule{20mm}{0mm}
\left(p_\mathrm{Kr} + p_\mathrm{Ks} + \white_\mathrm{Kr} +  \draw\right)^{99}
\left(p_\mathrm{Kr} + p_\mathrm{Ks} + \white_\mathrm{Ks} +  \draw\right)^{94}
$}}.
\end{equation}

In the likelihood function above we see, for example, that Anand won a
total of $18+6=24$ games playing white, $13+2=15$ games playing Black,
and that Anand played white against Karpov a total of $18+20+5=43$
times.  We fail to reject the null of
$\white_\mathrm{An}=\white_\mathrm{Kr}=\white_\mathrm{Ks}$, having a
likelihood ratio of $4.5$, or an asymptotic $p$-value of $0.076$.
There is no strong evidence that the players differ in regard to their
white advantage.


\section{Collusion in the World Chess Championship 1963}

The Method of Support allows one to estimate proclivity to draw while
accounting for players' strengths and first-move advantage.  We now
consider one infamous incident, following the World Chess
Championships 1963, in which Fischer famously and publicly claimed
that three of the Soviet players colluded.  The accusation was that
Keres, Petrosian, and Geller agreed ahead of time to draw when playing
one another, in order to conserve energy for subsequent matches.

\textcolor{blue}{\citet{moul2009} consider this accusation from a
  statistical perspective.  Their analysis incorporated a dataset of
  7355 games played in the period 1940-1978, used to calibrate the
  ability of the players.  In contrast, the analysis presented here
  uses only game outcomes (win-loss-draw) in the competition itself.}

 \begin{table}
 \caption{Cura\c{c}ao 1962 \label{candidates} Candidates, World Chess Championship 1963}
 \fbox{%
 \begin{tabular}{ll|cccccccc}  &
 &%\rotatebox{-90}{Petrosian} 
 &\rotatebox{-90}{Keres} 
 &\rotatebox{-90}{Geller}
 &\rotatebox{-90}{Fischer} 
 &\rotatebox{-90}{Korchnoi}
 &\rotatebox{-90}{Benko}
 &\rotatebox{-90}{Tal}
 &\rotatebox{-90}{Filip}\\
 Petrosian &USSR    &             &DDDD    &DDDD   &DWDD   &DDWW   &DDWD   &WWD    &DWWD\\
 Keres     &USSR    &             & -      &DDDD   &LDWD   &DDWD   &WWWL   &WDW    &DWWD\\
 Geller    &USSR    &             & -      & -     &WWDL   &DDWD   &DDDW   &DWW    &DWWD\\
 Fischer   &USA     &             & -      & -     & -     &LWLD   &LWDW   &DWD    &WDWD\\
 Korchnoi  &USSR    &             & -      & -     & -     & -     &DDDL   &WLD    &WWWW\\
 Benko     &USA     &             & -      & -     & -     & -     & -     &WLD    &LWWD\\
 Tal       &USSR    &             & -      & -     & -     & -     & -     & -     &WLD \\
 Filip     &TCH     &             & -      & -     & -     & -     & -     & -     & -  \\
 \end{tabular}}
 \end{table}

Here I consider game results from Cura\c{c}ao 1962, reproduced in
Table~\ref{candidates}.  We see that all twelve of the games between
the three accused players ended in a draw, a significant result given
51 winning games and 54 draws (Fisher's exact test, one-sided,
$p=0.0079$).  To quantify the evidence for any such collusion,
accounting for the player skills, we introduce a new draw entity: a
collusive draw monster with strength \cdraw, who wins when two
colluding players draw, and is inactive whenever a game has at least
one non-colluding player.  Then $H_N\colon\cdraw=\draw$ corresponds to
no collusion and the likelihood function would have

\begin{equation}\parbox{4in}{$
P^5\,
{Ke}^5\,
G^4
\mathit{Fis}\,  % Fischer, cf Filip
\mathit{Ko}^5
B\,
T^2  % Tal
\mathit{Fil}\,
\draw^{42}\,\cdraw^{12}\oo\\ \rule{10mm}{0mm}
\left(P+\white\right)^3
\left(\mathit{Ke}+\white\right)^4
\left(G+\white\right)^4\
\left(\mathit{Fis}+\white\right)^7\oo\\ \rule{20mm}{0mm}
\left(\mathit{Ko}+\white\right)^2
\left(B+\white\right)^5
\left(T+\white\right)
\left(\mathit{Fil}+\white\right)\oo
$}\end{equation}

\noindent for the numerator (where \cdraw\ is the strength of a collusive
draw), and a total of 30 terms such as
$\left(P+\mathit{Ke}+\white+\cdraw\right)^{4}$ [indicating that Petrosian played
  Keres 4 times] in the denominator.  The likelihood ratio for the
hypothesis that collusive draws have the same strength as regular
draws is about $e^{11.8}\simeq 10^5$: we conclude that the high number
of draws among the three accused players was unlikely to be due to
chance.

One plausible explanation for this result is that Soviet players have
a higher proclivity to draw than western players.  We can investigate
this by considering the interzonal tournament which took place in
Stockholm in 1962.

\subsection{Soviet playing style in the interzonal Stockholm 1962}

\newcommand{\halfw}{$\nicefrac{1}{2}$}
\newcommand{\halfb}{\colorbox{black!65}{\makebox[2em]{\strut\textcolor{white}{$\nicefrac{{\bf 1}}{{\bf 2}}$}}}}
\newcommand{\zerow}{$0$}
\newcommand{\zerob}{\colorbox{black!65}{\makebox[2em]{\strut\textcolor{white}{$0$}}}}
\newcommand{\oneww}{$1$}
\newcommand{\onebb}{\colorbox{black}{\textcolor{white}{$1$}}}



\begin{table}
\caption{Subset of the Fifth Interzonal round-robin, Stockholm
  1962.\label{interzonal} Results $\in\{0,\halfw,1\}$ and box colour
  refer to row player.  Thus, for example, Fischer played black
  against Geller and drew. 
  The full dataset, not shown here, has 23
  players}
\centering
\fbox{%
\begin{tabular}{l|cccccc}
&\rotatebox{-90}{Fischer} 
&\rotatebox{-90}{Geller} 
&\rotatebox{-90}{Petrosian\ }
&\rotatebox{-90}{Korchnoi} 
&\rotatebox{-90}{Filip}
& $\cdots$
\\ \hline %       Fischer   Geller    Petrosian Korchnoi    Filip
Fischer    (USA)  &   -     & \halfb  & \halfw  & \oneww    &\halfb    &\\
Geller    (USSR)  & \halfw  &    -    & \halfb  & \halfw    & \halfw   &\\
Petrosian (USSR)  & \halfb  & \halfw  &    -    & \halfb    & \halfw   &\\
Korchnoi  (USSR)  & \zerob  & \halfb  & \halfw  &   -       & \oneww   &\\
Filip      (TCH)  & \halfw  & \halfb  & \halfb  & \zerob    &  -       &\\
$\vdots$&&&&&\\
\end{tabular}}
\end{table}

The fifth Interzonal took place in Stockholm in 1962, in the form of a
23-player round-robin tournament.  For the 20 game results shown in
Table~\ref{interzonal}, we would have a likelihood function

\begin{equation}
%\mathcal{L}\left(F,G,P,K,L,\white,\draw\right)=
%                D/(F+G+W+D)  (F+P+D)/(F+P+D+W)   (F+W)/(F+W+D+K)   D/(F+W+L+D)
%                             D/(G+P+D+W)         D/(G+K+W+D)       D/(G+F+W+D)
%                                                 D/(P+K+W+D)       D/(P+L+W+D)
%                                                                   (K+W)/(K+L+W+D)
\frac{
  \draw^8(F+\white)(K+\white)
}{\parbox{4in}{$
    (F+G+\white+\draw)(F+P+\white+\draw)\\ \rule{10mm}{0mm}(F+K+\white+\draw)  (F+L+\white+\draw)\\ \rule{20mm}{0mm}
         (G+P+\white+\draw)  (G+K+\white+\draw)\\ \rule{30mm}{0mm}(G+F+\white+\draw)
                    (P+K+\white+\draw)\\ \rule{40mm}{0mm}(P+L+\white+\draw)
                               (K+L+\white+\draw)
 $}}
\end{equation}

\noindent (the numerator includes 8 draws and 2 wins, one for Fischer
and one for Korchnoi, both playing white); but the full likelihood
function for the interzonal has 297 terms, including $\draw^{110}$.
To investigate the possibility of increased Soviet drawing proclivity,
we introduce a Soviet draw monster with strength~$\draw_s$, who wins
when two Soviet players draw, and is inactive if a game has at least
one non-Soviet player.  Then $H_N\colon \draw_s=\draw$ corresponds to
a null of Soviet players having the same drawing proclivity as
non-Soviets.  The likelihood function for the data of
Table~\ref{interzonal} would be

\begin{equation}\label{collusion}
%\mathcal{L}\left(F,G,P,K,L,\white,\draw,\draw_s\right)=
%                D/(F+G+W+D)  (F+P+D)/(F+P+D+W)   (F+W)/(F+W+D+K)   D/(F+W+L+D)
%                             D/(G+P+D+W)         D/(G+K+W+D)       D/(G+F+W+D)
%                                                 D/(P+K+W+D)       D/(P+L+W+D)
%                                                                   (K+W)/(K+L+W+D)
\frac{
  \draw_s^3\draw^5(F+\white)(K+\white)
}{\parbox{4in}{$    (F+G+\white+\draw)(F+P+\white+\draw)\\ \rule{10mm}{0mm}(F+K+\white+\draw)  (F+L+\white+\draw)\\ \rule{20mm}{0mm}
         (G+P+\white+\draw_s)  (G+K+\white+\draw_s)\\ \rule{30mm}{0mm}(G+F+\white+\draw)
                    (P+K+\white+\draw_s)\\ \rule{40mm}{0mm}(P+L+\white+\draw)
                               (K+L+\white+\draw)
 $}}.
\end{equation}


The maximum likelihood estimate for the full 23-player dataset is
$\draw\simeq 0.043, \draw_s=0.354$; a constrained maximization gives a
likelihood ratio of about 5096, or a $p$-value of about $3.6\times
10^{-5}$: there is strong evidence that Soviet players do indeed draw
more frequently than non-Soviets.  Indeed we could go further and plot
a profile likelihood curve for the log-odds ratio
$\log\left(\draw_s/\draw\right)$ which, if Dirichlet, would be
accurately parabolic over a wide range~\cite[page 343]{ohagan2004}.
This is illustrated in figure~\ref{proflike}, that shows a credible
interval of about 1.8-2.4: we can be reasonably sure that
$\draw_s/\draw\geqslant e^{1.8}\simeq 6.05$.


\begin{figure}[htbp]
\begin{center}
\makebox{\includegraphics{collusion_chess_draws-proflike.pdf}}
\caption{Profile likelihood \label{proflike} for odds ratio
  $\log\left(\draw_s/\draw\right)$: note the accurately parabolic figure in
  the region plotted}
  \end{center}
\end{figure}



\textcolor{DarkGreen}{It is reasonable to suppose that collusion of
  the type considered here is ``effective'' in the sense that
  noncollusive games are detrimental to future play in a way that
  collusively drawn games are not.  At the end of the tournament,
  Petrosian, Keres and Geller had each played a total of 18
  noncollusive matches while Fischer, Benko and Filip had played a
  total of 26 or 27 (Tal, retiring mid tournament due to illness,
  played only 21).  For each game in the tournament, we count the
  total noncollusive games played up to that point for both the white
  and black player.  If, in that game, the number of previous
  noncollusive games for the white and black player differs by more
  than one\footnote{\textcolor{DarkGreen}{If collusivity is ignored,
    no game in the tournament is played between two players with
    cumulative game count differ by more than one.}}, we posit a ``rest
  monster'' whose strength helps the player who has played fewer
  noncollusive games.}

\textcolor{DarkGreen}{
  Observing that the rest monster is
  active in 49 of the 105 games of the tournament, we may consider
  these 49 games and, discarding draws, test whether rested players
  actually win more often than expected; observations are given in
  Table~\ref{restmonsterfisher} .  We reject the null of rested
  players having the same strength as their more tired opponents
  (one-sided Fisher's exact test, $p\simeq 10^{-4}$).}

\begin{table}
\textcolor{DarkGreen}{
  \caption{Results for the 25 won games (out of 49 played) in
    Cura\c{c}ao 1962 between rested and unrested
    players \label{restmonsterfisher}} \centering \fbox{%
    \begin{tabular}{l|cc}\\
      & White rested & Black rested \\ \hline
      White wins  & $9$   & $ 3$   \\
      Black wins  & $0$   & $13$   \\
  \end{tabular}}
  }
\end{table}

\textcolor{DarkGreen}{The Bradley-Terry model together with a white
  player advantage, draw monster, and rest monster, estimates the
  strength of the rest monster at about 26\% (compare the draw monster
  at 24\% and the White advantage at about 3\%).  The null of the rest
  monster having zero strength is rejected with a likelihood ratio of
  about $e^{11.9}$ or a $p$-value of about $10^{-6}$.  It is an
  interesting observation that neither Table~\ref{restmonsterfisher},
  nor reified Bradley-Terry, admit a White advantage (one-sided
  McNemar test, $p=\frac{1}{8}$, likelihood ratio $1.609$
  respectively).  }

\section{Conclusions and further work}

The present work details an approach to chess analysis based on
reification of entities representing proclivity to draw, first-mover
(or white player) advantage, and \textcolor{DarkGreen}{the additional
  playing strength conferred by previous collusive draws.  Likelihood
  functions for the players' and reified entities' strengths are
  defined using only specific tournament outcomes.}  Many hypotheses
of interest to the chess community may be analysed straightforwardly.

In this context we note, following the observation of Moul and Nye
that a player may well adopt a riskier strategy after a sequence of
defeats, adopting a ``nothing-to-lose'' mentality.  Such a phenomenon
would be detectable, at least in principle, by generalizing the
personalised draw likelihood function
\ref{KKA_personalised_drawmonster} and seeking evidence for the draw
strength changing as a tournament progresses: it would become stronger
for a player on a winning streak, and weaker for a losing player.

\textcolor{red}{One strong assumption made in the
  Kasparov-Karpov-Anand analysis was the constancy of the players'
  strengths over a 36-year period.  One approach might be to consider
  only matches played over a shorter span, but this would result in a
  smaller and less informative dataset; the tension between timeliness
  and informativeness is a delicate and difficult matter.}

\textcolor{orange}{In a wider context, many sports have a career arc
  whereby competitors start their careers weak, become stronger with
  experience, and finally become uncompetitive due to age.  Further
  work might include the introduction of additional reified entities
  that account for such phenomena: we might have a ``mid career
  monster'' who helps competitors at or near their career peak.}

\textcolor{orange}{The maximum likelihood estimate for strengths
  furnishes a natural ranking for the competitors.  \cite{lazear1981}
  observe that such ranking systems occur naturally in organizational
  situations (tournament theory) and the analysis presented here might
  be applicable to chess, with the advantage of allowing for
  both draws and incumbent advantage.}
 
\cite{moul2009} present detailed Monte Carlo simulations to assess the
likely impact of a putative Soviet cartel on tournament outcomes.  In
principle, the present system of likelihood and reified entities would
admit such simulations and it might be interesting to compare results.
\textcolor{blue}{\cite{moul2009} go on to compare round robin
  tournaments to knockout format events, searching for evidence of a
  Soviet draw cartel.  The analysis presented here, only considering
  international tournaments [where Soviets have incentive to play
    differently against fellow Soviets or Westerners] does not admit
  this phenomenon, but the technique could easily be applied to a more
  informative dataset.}

\textcolor{DarkGreen}{The Bradley-Terry techniques used here furnish
  evidence that collusive games played during Cura\c{c}ao 1962 were
  less detrimental to future performance than noncollusive games.
  However, no cognitively realistic model for mental
  fatigue~\citep{boksem2008,linden2003} was used.  There was no
  allowance for recovery, for example; and the criterion for
  restedness itself was quite coarse.  It would be possible in
  principle to investigate such mechanisms using further reified
  entities.}


\subsection*{\textcolor{DarkGreen}{Acknowledgement}}

\textcolor{DarkGreen}{The ``rest monster'' concept is due to a
  suggestion by an insightful JEBO referee}

\subsection*{Computational resources, supplementary material}

R code for all statistical analysis given here is available at

{\tt https://github.com/RobinHankin/hyper2}.

\bibliography{jebo_chess.bib}

\end{document}
