\documentclass[12pt]{article}
\usepackage{xcolor}
\begin{document}

\section*{MS596 ``Analysis of competitive surfing tournaments with generalized Bradley-Terry likelihoods" by Driver and Hankin: rebuttal}

Below, the reviewer's comments are in black, and our replies to the
issues are in \textcolor{blue}{blue}.  We have indicated changes to
the manuscript where appropriate.

Short story: we have accommodated all the comments with rewording and
clarification.  The resulting document is, I believe, stronger and
more scholarly than before and I recommend it to you.

\section*{Detailed rebuttal: Reviewer \#1}

To the authors,

An interesting paper submitted, thank you. My only
real concerns are your actual comments related to the sport of
competitive surfing.

Firstly, some confusion as you initially state heats involve one or
two other competitiors. However, you begin the second paragraph in the
introduction with a statement saying 'up to four surfers', so which is
it?

{\bf intent: } we should remove "up to four surfers"


Second statement is that the World Surf League is not the main
governing body of surfing competition. What about the International
Surfing Association, who organised and ran the Olympics, also run the
World Surfing games and national level competitions. The WSL really
only run professional level competition.

{\bf intent: }  WSL is the *professional* body organisation.  


34 surfers are for the mens competition.


{\bf intent: } We focus on the male competition for the moment

Why are you ignoring the "wildcards". No real justification for
this. Especially as some of the wildcards competed in more than one of
the competitions. So some form of statement to validate why you omit
the wildcards would be good.

One other comment is based around the statements around judging in the
second last paragraph before section 1.1. Judging is based on the
heats conditions, as surfing is performed within an ever changing
environment. Therefore, to try and compare between surfers throughout
a competition is really not applicable. As each heat, and each wave is
different from the previous. So a direct comparison throughout a
competition is somewhat unfair for the competitors. As for the claim
around abilities, is this not subjective, as each performer is judged
upon their surfing prowess on that particular wave in that particular
heat.


{\bf intent:}  Will put some statistical comment in here.



Your hypothesis on Brazilians preference for reef breaking waves is based on what? 


{\bf intent: } We tested for Brazilians NOT liking reef breaking waves
and liking point and beach breaking waves.  consistent with 6.1


Statement on the points system not affecting competitiors
behaviour. How could you say this, as depending on the finishing spot
at each competition, the athlete is awarded a set number of points. So
a higher placing in a competition will award you a greater number of
points. Impacting upon your placing in the rankings for the subsequent
competition. And therefore, your seeding for the next competition.



{\bf intent: } need to rephrase


Other than that, an interesting paper. Just feel you need to clarify a
few of the points above to provide readers a clearer viewpoint of the
sport and understanding to the importance of your study. As I found it
quite interesting. Just some claims and anomalies related to
competitive surfing.
  

\section*{Detailed rebuttal: Reviewer \#2}


The World Surf League (WSL) is the main governing
body of surfing competitions (World Surf League 2021).

{\bf intent}  dealt with above

Not exactly. What about ISA (world surfing games, Olympics, national
championships, regional championships).

{\bf intent}  ditto


The WSL conducts a
world tour in which the best ranked 34 surfers compete;

This only constitutes for male competition

{\bf intent}  sorted above



in addition, at each tour venue, two "wildcard" surfers also enter the
competition, who are ignored here

Why are they ignored? Especially as some of the wildcards competed in multiple events.


{\bf intent}  sorted above.

Up to four surfers--the competitors--are in the water simultaneously,
watched by up to five judges.

You stated one or two other competitors above. So which is it?

However, for a particular heat, the order statistic--that is, which
competitor scored most highly, second highest, and third--is
informative about competitors abilities: the wave environment is
common to each surfer

But judging is based on the heats conditions, as surfing is performed
within an ever changing environment. Therefore, to try and compare
between surfers throughout a competition is really not applicable. As
each heat, and each wave is different from the previous. So a direct
comparison is somewhat unfair for the competitors.



and here we apply it to assess the supposed Brazilian preference for
reef-breaking waves.

Based on what hypothesis?


the points system itself is intrinsically arbitrary: the details of
the points system does not affect the competitors' behaviour but can
change the overall ranking;

How could you say this, as depending on finishing spot at each
competition, the athlete is awarded a set number of points. So a
higher placing in a competition will award you a greater number of
points. Impacting upon your placing in the rankings for the subsequent
competition. And therefore, your seeding for the next competition.
\end{document}
