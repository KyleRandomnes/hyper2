\documentclass[article]{jss}


\author{Robin K. S. Hankin\\Auckland University of Technology}
\title{Generalized Plackett-Luce Likelihoods}
\Plainauthor{Robin K. S. Hankin} %% comma-separated
\Plaintitle{Generalized Plackett-Luce Likelihoods}

%% an abstract and keywords
\Abstract{
  The \code{hyper2} package is improved here to include \code{hyper3}
  objects.
}
\Keywords{Plackett-Luce, Bradley-Terry, Mann-Whitney}
\Plainkeywords{Plackett-Luce, Bradley-Terry, Mann-Whitney}

%% publication information
%% NOTE: Typically, this can be left commented and will be filled out by the technical editor
%% \Volume{50}
%% \Issue{9}
%% \Month{June}
%% \Year{2012}
%% \Submitdate{2012-06-04}
%% \Acceptdate{2012-06-04}

%% The address of (at least) one author should be given
%% in the following format:
\Address{
  Robin K. S. Hankin\\
  School of Engineering and Mathematical Sciences\\
  Auckland University of Technology\\
  Wellesley Street\\
  Auckland, New Zealand\\
  E-mail: \email{hankin.robin@gmail.com}\\
  URL: \url{https://academics.aut.ac.nz/robin.hankin}
}

%% for those who use Sweave please include the following line (with % symbols):
%% need no \usepackage{Sweave.sty}

%% end of declarations %%%%%%%%%%%%%%%%%%%%%%%%%%%%%%%%%%%%%%%%%%%%%%%


\begin{document}

Lorem ipsum dolor sit amet, consectetur adipiscing elit, sed do
eiusmod tempor incididunt ut labore et dolore magna aliqua.  Ut enim
ad minim veniam, quis nostrud exercitation ullamco laboris nisi ut
aliquip ex ea commodo consequat.  Duis aute irure dolor in
reprehenderit in voluptate velit esse cillum dolore eu fugiat nulla
pariatur.  Excepteur sint occaecat cupidatat non proident, sunt in
culpa qui officia deserunt mollit anim id est laborum

The \code{hyper2} package~\cite{hankin2017} furnishes computational
support for generalized Plackett-Luce~\citep{plackett1975} likelihood
functions.  Our preferred interpretation is a race (as in track and
field atheletics): given six competitors $1-6$, we ascribe to them
nonnegative strengths $p_1\ldots p_6$; the probability that $i$ beats
$j$ is $p_i/(p_i+p_j)$.  It is conventional to normalise so that the
total strength is unity, and to identify a competitor with his
strength.  Given an order statistic, say $p_1\succ p_2\succ p_3\succ
p_4$, the Plackett-Luce likelihood function would be

\begin{equation}
  \frac{p_1}{p_1+p_2+p_3+p_4}\cdot
  \frac{p_2}{    p_2+p_3+p_4}\cdot
  \frac{p_3}{        p_3+p_4}\cdot
  \frac{p_4}{            p_4}.
\end{equation}

The \code{hyper2} package generalized this likelihood function to
functions of ${\mathbf p}=(p_1,\ldots,p_n)$ with

\begin{equation}\label{hyper2likelihood}
\mathcal{L}\left(\mathbf{p}\right)=
\prod_{s\in \mathcal{O}}\left({\sum_{i\in s}}p_i\right)^{n_s}
\end{equation}

\noindent
where~$\mathcal{O}$ is a set of observations and~$s$ a subset
of~$\left[n\right]=\left\{1,2,\ldots,n\right\}$; numbers~$n_s$ are
integers which may be positive or negative.  The approach adopted by
the~\pkg{hyperdirichlet} package~ is to sptore each of the~$2^n$
possible subsets of~$\left[n\right]$ together with an exponent:





\end{document}
