\documentclass[12pt]{article}
\usepackage{xcolor}
\begin{document}

\section*{An objective assessment of point scoring systems in Formula 1 motor racing: rebuttal}


\textcolor{blue}{
Below, I reproduce the reviewers comments; my writing, including
replies to the reviewer issues are in blue.  I have indicated changes
to the manuscript where appropriate.\\ \\
Short story: I have accommodated all the comments with rewording and
clarification.  The resulting document is, I believe, stronger and
more scholarly than before and I recommend it to you.
}

\subsection*{Detailed response to comments}

Conclusion in Introduction: The final paragraph of section 1
(Introduction) seems to need a conclusion.

Points systems similar to that of Formula 1 are common in other racing
sports; all have the common feature of translating ranks into points
which combine additively to generate an overall ranking.  Further, we
see points systems used in the wider context of competitive situations
such as the Eurovision Song Contest and many other such group
tournaments.  \mbox{Therefore, \ldots}\\ \\

\textcolor{blue}{\noindent I have added a sentence to the paragraph as
  suggested, which now reads:\\ \\ \\Points systems similar to that of
  Formula 1 are common in other racing sports; all have the common
  feature of translating ranks into points which combine additively to
  generate an overall ranking. Further, we see points systems used in
  the wider context of competitive situations such as the Eurovision
  Song Contest and many other such group tournaments.  {\bf Therefore,
    the ideas used here for analysis of motorsports are directly
    applicable to a broad range of competitive situations in which
    points are used to rank competitors.}}\\ \\

Addition of literature review of Zipf (1949)

Many of our readers are unfamiliar with this work.  Thus, a short
passage will help in understanding the significance of the inclusion
of Zipf in the manuscript.

\textcolor{blue}{I have realised that referring to a points system
  $\left(1,\frac{1}{2},\frac{1}{3},\frac{1}{4},\ldots\right)$ as
  ``Zipf's law'' is not really appropriate: I am not using it as a
  probability mass function but rather as a simple monotonically
  decreasing sequence.  I have removed references to ``Zipf's law'' in
  favour of the more descriptive ``Zipfian''.  I have added a sentence
  to this effect in the manuscript}\\ \\

The charts need to be larger for easier reading.  Please place figures
in full-page format as appendices.  Share the actual charts rather
than pictures of the charts.  Doing so will allow us to retain image
quality as figures are enlarged.

\textcolor{blue}{Charts now enlarged, also original PDFs supplied as
  requested}\\ \\

Replace Pie Charts with Pareto Charts

The pie charts are difficult to distinguish between components. For
example, in Figure 1 for 2017, the reviewers could not see a
difference between Bottas and Vettel.  The Pareto Chart style would
improve that aspect.

\textcolor{blue}{Done.  Pie charts replaced with Pareto charts}\\ \\


\section*{Rankings in Section 2.1}

Place the comparisons of the rankings systems in a table.  Readers will
be able to make comparisons more easily.

\textcolor{blue}{Done.}\\ \\

Table 2 and Figures 4-6 Consistency

Please use WTA or Top1 consistently in these items.

\textcolor{blue}{Done, wta used consistently now}\\ \\

Section 2.2 Numerical Results

Explain how the WTA system maximizes the probability of the complete
order statistic.

\textcolor{blue}{Brief sentence added summarizing the comparison
  method}\\ \\


Comment for Practitioners

Please include a summary of how the points system could benefit the
managers of Formula 1 and other series in the conclusion.

\textcolor{blue}{Summary added to the conclusions section.}\\ \\

Overall, your analysis of Formula 1 points systems and the comparison
of points systems to an objective rank ordering has the potential to
be a valuable resource for practitioners in the motorsports industry,
helping to inform strategy, evaluate fairness, and guide
decision-making.

\textcolor{blue}{Sentence added at the end of the document to this
  effect, slightly reworded from the reviewer's suggestion to fit
  better in context.}


\end{document}


