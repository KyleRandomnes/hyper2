\documentclass[12pt]{article}
\usepackage{xcolor}
\begin{document}

\section*{JSS4832: Generalized Plackett-Luce Likelihoods by Robin
K. S. Hankin}


Below, the reviewer's comments are in black, and the replies to the
issues are in \textcolor{blue}{blue}.  I have indicated changes to
the manuscript where appropriate.

Short story: I have accommodated all the comments with rewording and
clarification.

\subsection*{From the editorial team:}

The manuscript currently lacks a thorough software overview allowing
the reader to assess which software packages (for R but also other
software environments) are available providing related functionality
and indicating how the presented functionality provides additional
functionality.

\textcolor{blue}{Response: A more detailed review of existing software
  is provided, although the key innovation (which is Plackett-Luce
  likelihood functions including competitors of identical
  Bradley-Terry strength) is not available elsewhere.}

The manuscript could benefit from some restructuring to better guide
the reader on available methods and implementations, the statistical
methodology covered and extended and the specific computational
implementation made.  It would seem beneficial to entangle these
different aspects as different readers might have differnt foci of
interest and could thus better select on which parts of the manuscript
to focus.  The sections also currently seem to be a mix between
different methods and illustrative applications. Again a thorough
overview on the statistical functionality provided by the package in a
separate section might ease accessibility for readers. Disussing
specific computational aspects of the implementation in a separate
section is certainly also of interest.

\textcolor{blue}{Response}


Note that in R many functions providing ML estimates do not only
return point estimates but either an object where then in an
additional step, e.g., the Hessian can be determined as well as
further information provided using summary / print etc., or directly
have already an object returned which contains more information on the
fit including also the log-likelihood attained, the model fitted, etc.

\textcolor{blue}{Response:}

As an example, methods(class = "PLADMM") in the PlackettLuce package
lists: anova, deviance, fitted, itempar, logLik, predict, print,
summary, vcov. The anova/fitted/predict might be specific to the
regression case but the others might be useful for hyper3 as
well. Similarly for btmodel() in the psychotools package and the eba
package while BradleyTerry2, BTLLasso, PLMIX, and prefmod provide
fewer methods.

The NAMESPACE contains

\verb=exportPattern("^[[:alpha:]]+")=

Please avoid this and check if this can be removed or how it could be
dropped.

It would seem that currently functions are exported which are not
supposed to be directly called by the user, e.g., no argument
checking. In this way they could be kept internal and documentation if
desired could be provided with keyword internal to avoid cluttering
the output of

\verb+help(package = "hyper2")+

\textcolor{blue}{Response:}

Manuscript style comments:

Code should have enough spaces to facilitate reading.  Please include
spaces before and after operators and after commas (unless spaces have
syntactical meaning).

\textcolor{blue}{Response:}

The rule for capitalizing the starting letters of Figure, Section and
Table is as follows: If you are referring to a particular
figure/section/table then capitalize the first letter, otherwise use a
lower-case first letter. For example, something shown in Section 4
vs. there are three sections in this paper.

\textcolor{blue}{Response:}

In all cases, code input/output must fit within the normal text
width of the manuscript.  Thus, code input should have appropriate
line breaks and code output should preferably be generated with a
suitable width (or otherwise edited).

\textcolor{blue}{Response:}

As a reminder, please make sure that: - \verb+\proglang, \pkg+ and
\verb+\code+ have been used for highlighting throughout the paper
(including titles and references), except where explicitly escaped.

\textcolor{blue}{Response:}

References:



o Please make sure that all software packages are \cite{}'d properly.

o All references should be in title style.

o See FAQ for specific reference instructions.


Code:

o Please make sure that the files needed to replicate all
code/examples within the manuscript are included in a standalone
replication script.


\textcolor{blue}{We believe that this referee's insightful and
  constructive comments have led to a much improved submission.}

\end{document}

